%% Generated by Sphinx.
\def\sphinxdocclass{report}
\documentclass[a4paper,11pt,spanish]{sphinxmanual}
\ifdefined\pdfpxdimen
   \let\sphinxpxdimen\pdfpxdimen\else\newdimen\sphinxpxdimen
\fi \sphinxpxdimen=.75bp\relax
\ifdefined\pdfimageresolution
    \pdfimageresolution= \numexpr \dimexpr1in\relax/\sphinxpxdimen\relax
\fi
%% let collapsible pdf bookmarks panel have high depth per default
\PassOptionsToPackage{bookmarksdepth=5}{hyperref}
%% turn off hyperref patch of \index as sphinx.xdy xindy module takes care of
%% suitable \hyperpage mark-up, working around hyperref-xindy incompatibility
\PassOptionsToPackage{hyperindex=false}{hyperref}
%% memoir class requires extra handling
\makeatletter\@ifclassloaded{memoir}
{\ifdefined\memhyperindexfalse\memhyperindexfalse\fi}{}\makeatother

\PassOptionsToPackage{booktabs}{sphinx}
\PassOptionsToPackage{colorrows}{sphinx}

\PassOptionsToPackage{svgnames}{xcolor}

\PassOptionsToPackage{warn}{textcomp}

\catcode`^^^^00a0\active\protected\def^^^^00a0{\leavevmode\nobreak\ }
\usepackage{cmap}
\usepackage{fontspec}
\defaultfontfeatures[\rmfamily,\sffamily,\ttfamily]{}
\usepackage{amsmath,amssymb,amstext}
\usepackage{polyglossia}
\setmainlanguage{spanish}



\setmainfont{DejaVu Serif}
\setsansfont{DejaVu Sans}
\setmonofont{DejaVu Sans Mono}



\usepackage[Bjornstrup]{fncychap}
\usepackage{sphinx}
\sphinxsetup{
    InnerLinkColor={RGB}{240,56,97},
    % TitleColor={RGB}{240,56,97},  % Comentado para evitar aplicar color global al título
    verbatimwithframe=false,
    VerbatimColor={rgb}{0.9,0.9,0.9},
    VerbatimBorderColor={rgb}{0.8,0.8,0.8},
    }
\fvset{fontsize=\small}
\usepackage{geometry}


% Include hyperref last.
\usepackage{hyperref}
% Fix anchor placement for figures with captions.
\usepackage{hypcap}% it must be loaded after hyperref.
% Set up styles of URL: it should be placed after hyperref.
\urlstyle{same}

\addto\captionsspanish{\renewcommand{\contentsname}{Contenidos:}}

\usepackage{sphinxmessages}
\setcounter{tocdepth}{2}


% Paquetes adicionales
\usepackage[firstpage=true]{background}
\usepackage{graphicx}
\usepackage{titling}
\usepackage{etoolbox}
\usepackage[titles]{tocloft}

% Definir la ruta de búsqueda para imágenes
\graphicspath{{_static/images/}}

% Configuración de la imagen de fondo para la primera página
\backgroundsetup{
  scale=1,
  color=black,
  opacity=0.3,
  angle=0,
  contents={\includegraphics[width=\paperwidth,height=\paperheight]{cabecerakan.png}}
}

% Ocultar autor y fecha, asegurando que no se muestren
\preauthor {}  % Vaciar autor
\postauthor {}  % Sin efecto de autor
\predate {}  % Sin fecha
\date {}
\postdate {}  % Sin fecha


% Definir la macro de versión a partir del valor en conf.py
\newcommand{\therelease}{2.2.0}

    % Redefinir \sphinxmaketitle para ajustar la posición y formato del título
    \renewcommand{\sphinxmaketitle}{%
      \begingroup
        \begin{center}
          \vspace*{4cm}  % Mover el título hacia abajo
          {\Huge \textbf{\color[RGB]{240,56,97} \thetitle}}  % Título del proyecto en grande y color magenta
          \vspace{1.5cm}  % Espacio entre el título y la versión
          \\
          {\Large \textbf{\color[RGB]{240,56,97} Versión:} \color[RGB]{240,56,97} \therelease}  % Mostrar la versión en grande y en color magenta
        \end{center}
      \endgroup
    }


% Configuración de la tabla de contenidos y estilos
\cftsetpnumwidth{1.25cm}
\cftsetrmarg{1.5cm}
\setlength{\cftchapnumwidth}{0.75cm}
\setlength{\cftsecindent}{\cftchapnumwidth}
\setlength{\cftsecnumwidth}{1.25cm}
\renewcommand{\cftsecpagefont}{\color{red}}

% Eliminar numeración de las páginas previas a los capítulos
    \fancypagestyle{plain}{ % Definir estilo para páginas previas al capítulo
        \fancyhf{}  % Eliminar encabezado y pie de página
        \renewcommand{\headrulewidth}{0pt}  % Eliminar línea del encabezado
        \renewcommand{\footrulewidth}{0pt}  % Eliminar línea del pie de página
        \pagestyle{empty}  % No numerar las páginas
    }

    % Configurar estilo de las páginas de los capítulos
    \fancypagestyle{normal}{ % Estilo para las páginas de los capítulos
        \fancyhf{}  % Eliminar cualquier encabezado y pie por defecto
        \renewcommand{\headrulewidth}{0pt}  % Eliminar la línea del encabezado
        \renewcommand{\footrulewidth}{0pt}  % Eliminar la línea del pie de página
        \fancyfoot[R]{\thepage}  % Numerar las páginas de los capítulos en el centro
    }

    % Asegurar que solo las páginas de los capítulos muestren la numeración en arábigos
    \renewcommand{\chapterpagestyle}{normal}  % Aplicar estilo 'normal' a las páginas de los capítulos
    \pagenumbering{arabic}  % Usar numeración arábiga para las páginas del capítulo

% Establecer color y alineación predeterminados para el contenido principal
    \AtBeginDocument{%
      \color{black}  % Establece el color del texto principal a negro
      \raggedright   % Alinea el texto a la izquierda sin justificar
    }



\title{GeoExpress Visor}
\date{15 de octubre de 2024}
\release{2.2.0}
\author{Kan Territory \& IT}
\newcommand{\sphinxlogo}{\vbox{}}
\renewcommand{\releasename}{Versión}
\makeindex
\begin{document}

\pagestyle{empty}
\sphinxmaketitle
\pagestyle{plain}
\sphinxtableofcontents
\pagestyle{normal}
\phantomsection\label{\detokenize{index::doc}}



\chapter{Introducción}
\label{\detokenize{index:introduccion}}
\sphinxAtStartPar
En este documento se describen las distintas funcionalidades y capacidades del visor de Geoexpress \sphinxurl{https://geoexpress.kan.com.ar}, en su versión 2.2.0
El objetivo es que el usuario final pueda navegar cómodamente y aprovechar todas las herramientas que ofrece. Para ello se realizará un repaso detallado de todas las secciones y botones del visor, haciendo énfasis en la perspectiva del usuario dentro de la interfaz.

\noindent\sphinxincludegraphics{{geoexpress}.png}


\chapter{Tabla de contenidos}
\label{\detokenize{index:tabla-de-contenidos}}
\sphinxstepscope


\section{Navegación}
\label{\detokenize{navigation/index:navegacion}}\label{\detokenize{navigation/index::doc}}
\sphinxstepscope


\subsection{Navegación de la interfaz}
\label{\detokenize{navigation/interfaz:navegacion-de-la-interfaz}}\label{\detokenize{navigation/interfaz::doc}}

\subsubsection{Controles}
\label{\detokenize{navigation/interfaz:controles}}
\sphinxAtStartPar
A continuación, se muestra una lista de controles que son de utilidad en el visor de Geoexpress.
\begin{itemize}
\item {} 
\sphinxAtStartPar
\sphinxstylestrong{F11}: activa o desactiva el modo pantalla completa en el navegador web.

\item {} 
\sphinxAtStartPar
\sphinxstylestrong{F5}: actualiza el navegador web.

\item {} 
\sphinxAtStartPar
\sphinxstylestrong{Ctrl + F5}: actualiza el navegador web y además borra la memoria caché.

\end{itemize}


\subsubsection{Navegación por mouse}
\label{\detokenize{navigation/interfaz:navegacion-por-mouse}}
\sphinxAtStartPar
Aquí se describen las funciones de cada botón del mouse en el visor al momento de navegar en el mapa:
\begin{itemize}
\item {} 
\sphinxAtStartPar
\sphinxstylestrong{Clic izquierdo sin soltar}: mueve el mapa de derecha a izquierda y de arriba a abajo.

\item {} \begin{description}
\sphinxlineitem{\sphinxstylestrong{Clic derecho sin soltar}:}\begin{itemize}
\item {} 
\sphinxAtStartPar
\sphinxstylestrong{Mover de arriba a abajo}: cambia el ángulo del mapa, permitiendo pasar de la vista 2D a la 3D y a vistas con ángulos intermedios.

\item {} 
\sphinxAtStartPar
\sphinxstylestrong{Mover de derecha a izquierda}: rota el mapa.

\end{itemize}

\end{description}

\item {} 
\sphinxAtStartPar
\sphinxstylestrong{Rueda del mouse}: aumenta el zoom (movimiento hacia adelante) y reduce el zoom (movimiento hacia atrás).

\end{itemize}

\sphinxstepscope


\subsection{Barra lateral \sphinxhyphen{} Sidebar}
\label{\detokenize{navigation/sidebar:barra-lateral-sidebar}}\label{\detokenize{navigation/sidebar::doc}}
\noindent{\sphinxincludegraphics[scale=0.25]{{barra}.png}\hspace*{\fill}}

\sphinxAtStartPar
A la izquierda de la pantalla se puede ver una barra lateral vertical que permite navegar entre categorías y subcategorías para activar o desactivar las capas predefinidas por el cliente desde \sphinxstylestrong{Layers Manager} y visualizarlas u ocultarlas.

\noindent{\hspace*{\fill}\sphinxincludegraphics[scale=0.25]{{barra1}.png}}

\sphinxAtStartPar
Cada ícono representa una \sphinxstylestrong{categoría} distinta de capas, agrupadas por una temática en particular. Al hacer clic en una categoría, la barra se expandirá hacia la derecha y se abrirá el \sphinxstylestrong{panel de capas}, donde se podrán visualizar las \sphinxstylestrong{subcategorías}.
Estas agrupan a las capas dentro de las categorías en torno a temáticas más específicas que las de las categorías o sub\sphinxhyphen{}temáticas.

\sphinxAtStartPar
Al hacer clic en una subcategoría, finalmente aparecerán las \sphinxstylestrong{capas}.

\noindent{\sphinxincludegraphics[scale=0.3]{{barra2}.png}\hspace*{\fill}}

\sphinxAtStartPar
Al cliquear en cada capa, esta se activará y podrá verse sobre el mapa. Al hacer esto, también se abrirá una \sphinxstylestrong{tarjeta de capa} a la derecha del panel de capas. Es probable que debamos hacer click en en el signo mayor, a la izquierda de la cruz para desplegarla.

\noindent{\hspace*{\fill}\sphinxincludegraphics{{barra3}.png}\hspace*{\fill}}

\sphinxAtStartPar
En la parte inferior de la barra lateral, tenemos un icono con forma de engranaje.

\noindent{\hspace*{\fill}\sphinxincludegraphics{{barra4}.png}\hspace*{\fill}}

\sphinxAtStartPar
Al apretarlo nos permite ver un enlace a una breve guia de las opciones de la pantalla

\noindent{\hspace*{\fill}\sphinxincludegraphics{{barra5}.png}\hspace*{\fill}}

\sphinxAtStartPar
Con el botón lenguaje podremos elegir en que idioma deseamos utilizar el visor de Geoexpress

\noindent{\hspace*{\fill}\sphinxincludegraphics{{barra6}.png}\hspace*{\fill}}

\sphinxstepscope


\subsection{Tarjeta de capa}
\label{\detokenize{navigation/cards:tarjeta-de-capa}}\label{\detokenize{navigation/cards::doc}}
\sphinxAtStartPar
La tarjeta de capa permite al usuario diversas operaciones relacionadas con la capa.


\subsubsection{Opacidad}
\label{\detokenize{navigation/cards:opacidad}}
\noindent\sphinxincludegraphics{{cards1}.png}

\sphinxAtStartPar
En Geoexpress, la opacidad de todas las capas es del 50\% por defecto, a menos que se haya determinado una opacidad específica previamente. Algunas capas, como los mesh, DEMs y objetos 3D, tienen opacidad fija y no se puede modificar.


\subsubsection{Zoom a capa}
\label{\detokenize{navigation/cards:zoom-a-capa}}
\noindent\sphinxincludegraphics{{cards2}.png}

\sphinxAtStartPar
El ícono de \sphinxstylestrong{Zoom a la capa} ajusta la vista del mapa a la extensión de la capa que se encuentra actualmente activa.


\subsubsection{Filtros}
\label{\detokenize{navigation/cards:filtros}}
\noindent\sphinxincludegraphics{{cards3}.png}

\sphinxAtStartPar
El botón \sphinxstylestrong{Filtros} permite hacer un filtrado de los datos de la capa. Como ejemplo, la capa Localidades esta organizada por provincias y por medio del filtro, podremos seleccionar algunas, todas o ninguna

\begin{sphinxadmonition}{warning}{Advertencia:}
\sphinxAtStartPar
RESTRICCIÓN: Si se supera el límite de 30 opciones para seleccionar, el filtro de selección no aparecerá en la tarjeta de capas.
\end{sphinxadmonition}


\subsubsection{Más operaciones}
\label{\detokenize{navigation/cards:mas-operaciones}}
\noindent\sphinxincludegraphics{{cards4}.png}

\sphinxAtStartPar
El botón \sphinxstylestrong{+} nos muestra mas opciones para trabajar con las capas. Nos permite
\begin{quote}
\begin{itemize}
\item {} 
\sphinxAtStartPar
Descargar la capa en formato .zip

\end{itemize}

\noindent\sphinxincludegraphics{{cards5}.png}
\begin{itemize}
\item {} 
\sphinxAtStartPar
Activar las referencias de los elementos de la capa

\end{itemize}

\noindent\sphinxincludegraphics{{cards6}.png}
\begin{itemize}
\item {} 
\sphinxAtStartPar
Ver informacion de la capa

\end{itemize}

\sphinxAtStartPar
Apretando en

\noindent\sphinxincludegraphics{{cards7}.png}

\sphinxAtStartPar
Nos muestra

\noindent\sphinxincludegraphics{{cards8}.png}
\end{quote}


\subsubsection{Icono BIM}
\label{\detokenize{navigation/cards:icono-bim}}
\sphinxAtStartPar
El ícono de BIM está disponible solo para capas de objetos 3D con archivos .ifc cargados.

\noindent\sphinxincludegraphics{{cards9}.png}

\sphinxAtStartPar
Al hacer clic sobre el ícono, permite analizar el objeto 3D y activar o desactivar sus elementos. Para volver al mapa, se hace clic en la flecha en la esquina superior izquierda.

\noindent\sphinxincludegraphics{{cards10}.png}


\subsubsection{Administracion de capas}
\label{\detokenize{navigation/cards:administracion-de-capas}}

\paragraph{Superposición de capas}
\label{\detokenize{navigation/cards:superposicion-de-capas}}
\sphinxAtStartPar
Si se activan múltiples capas, el orden se determina por el orden de activación, con la última capa sobreponiéndose a las demás

\noindent\sphinxincludegraphics{{cards11}.png}

\sphinxAtStartPar
Este orden puede modificarse arrastrando las tarjetas de capa en el listado. Esto generará que también se cambie el orden de superposición de las capas en el mapa.

\noindent\sphinxincludegraphics{{cards12}.png}

\sphinxAtStartPar
quedando..

\noindent\sphinxincludegraphics{{cards13}.png}


\paragraph{Deshabilitar capas}
\label{\detokenize{navigation/cards:deshabilitar-capas}}
\sphinxAtStartPar
Para desactivar una capa, se puede destildar en el panel de capas o cerrar la tarjeta de capa.

\noindent\sphinxincludegraphics{{cards14}.png}

\sphinxAtStartPar
La barra negra con la leyenda «Capas activas» permite desactivar todas las capas activas con el botón de cesto de basura.

\noindent\sphinxincludegraphics{{cards15}.png}

\sphinxAtStartPar
La barra de capas activas también posee una flecha para arriba que, al hacer clic sobre ella, oculta las tarjetas de capa. Una vez ocultas, puede hacerse clic en la misma flecha, ahora orientada hacia abajo, para mostrar de nuevo las tarjetas.

\noindent\sphinxincludegraphics{{cards16}.png}

\sphinxAtStartPar
Para maximizar la superficie del mapa en la pantalla, se puede cerrar el panel de capas haciendo clic en la cruz en la esquina superior derecha. Para reabrir el panel, se hace clic en alguna categoría.

\noindent\sphinxincludegraphics{{cards17}.png}

\sphinxstepscope


\section{Búsqueda}
\label{\detokenize{search/index:busqueda}}\label{\detokenize{search/index::doc}}
\sphinxstepscope


\subsection{Buscador}
\label{\detokenize{search/search:buscador}}\label{\detokenize{search/search::doc}}
\noindent{\sphinxincludegraphics[scale=0.5]{{search1}.png}\hspace*{\fill}}

\sphinxAtStartPar
El visor de Geoexpress cuenta con un buscador similar a los de Google Maps y Bing Maps, que sirve para encontrar ubicaciones exactas en el mapa. Para usarlo, se escribe el nombre del lugar en el buscador en la esquina superior derecha y se presiona Enter. Aparecerá una lista de opciones y, al seleccionar una, el mapa hará zoom en la ubicación elegida.

\sphinxstepscope


\subsection{Acceso directo a GeoNode}
\label{\detokenize{search/geonode:acceso-directo-a-geonode}}\label{\detokenize{search/geonode::doc}}
\sphinxAtStartPar
Este visor permite el acceso directo al GeoNode de Geoexpress de Kan \sphinxurl{https://geoexpress-demo.kan.com.ar} mediante el logo de Kan que se encuentra en la parte inferior izquierda del visor. Al hacer clic en el logo, se redireccionará al usuario al GeoNode.

\noindent{\hspace*{\fill}\sphinxincludegraphics{{search2}.png}\hspace*{\fill}}

\sphinxAtStartPar
Si contamos con los permisos suficientes podremos cargar nuevas capas, en caso contrario podremos consultarlas.

\noindent{\hspace*{\fill}\sphinxincludegraphics{{search3}.png}\hspace*{\fill}}

\sphinxstepscope


\section{Herramientas de navegación}
\label{\detokenize{tools/index:herramientas-de-navegacion}}\label{\detokenize{tools/index::doc}}
\sphinxstepscope


\subsection{Ubicación}
\label{\detokenize{tools/location:ubicacion}}\label{\detokenize{tools/location::doc}}
\sphinxAtStartPar
Para encontrar la ubicación del dispositivo del usuario, se utiliza el botón que se encuentra en la parte inferior izquierda de la pantalla, debajo del logo de Kan. Al hacer clic, se hará zoom automáticamente en la ubicación del usuario, representada con un punto azul y un margen de error mostrado con un círculo azul claro.

\noindent{\hspace*{\fill}\sphinxincludegraphics{{location}.png}\hspace*{\fill}}

\sphinxstepscope


\subsection{Zoom}
\label{\detokenize{tools/zoom:zoom}}\label{\detokenize{tools/zoom::doc}}
\noindent{\hspace*{\fill}\sphinxincludegraphics{{zoom1}.png}}

\sphinxAtStartPar
Para ajustar el nivel de zoom en el visor, use los botones \sphinxstylestrong{+} para acercar la vista del mapa y \sphinxstylestrong{\sphinxhyphen{}} para alejarla. El nivel de zoom se muestra en la URL del visor, después del código de idioma y el \sphinxstylestrong{\#}. Por ejemplo, en la URL \sphinxurl{https://geoexpress.kan.com.ar/es\#17.56/-34.608646/-58.372966}, el nivel de zoom es 17.56. El zoom máximo es 22, y el mínimo es 0.9.

\sphinxAtStartPar
Visualización a zoom 22

\noindent\sphinxincludegraphics[scale=0.25]{{zoom2}.png}

\sphinxAtStartPar
Visualización a zoom 0.9

\noindent\sphinxincludegraphics[scale=0.25]{{zoom3}.png}

\sphinxstepscope


\subsection{Orientación}
\label{\detokenize{tools/orientation:orientacion}}\label{\detokenize{tools/orientation::doc}}
\sphinxAtStartPar
En ocasiones, el mapa no tendrá el norte orientado hacia arriba, tal como se lo ve convencionalmente en los mapas, y entonces tendrá los puntos cardinales rotados. Para ayudar al usuario a orientarse en todo momento, el visor posee un ícono de brújula que cumple esa función. Como toda brújula, su aguja roja siempre apunta hacia el norte sin importar cuán rotado esté el mapa.

\noindent{\hspace*{\fill}\sphinxincludegraphics{{orientacion1}.png}\hspace*{\fill}}

\sphinxAtStartPar
Para que el mapa vuelva a tener el norte arriba, solo hace falta hacer clic sobre dicho ícono.

\noindent{\hspace*{\fill}\sphinxincludegraphics{{orientacion2}.png}\hspace*{\fill}}

\sphinxstepscope


\subsection{Vista 3D/2D}
\label{\detokenize{tools/3d-2d:vista-3d-2d}}\label{\detokenize{tools/3d-2d::doc}}
\sphinxAtStartPar
El visor Geoexpress ofrece la ventaja de poder mostrar capas tanto en 3D como en 2D. Para optimizar la visualización de cada tipo de capa, el visor incluye una vista 3D y una vista 2D, siendo esta última la visualización cenital predeterminada del mapa. Entre las herramientas de navegación, el ícono de vista 3D, al hacer clic, activa dicha vista, inclinando el mapa para permitir una mejor apreciación del relieve en capas 3D, ya sea de una capa vectorial, un objeto 3D, un DEM o un mesh.

\sphinxAtStartPar
Para volver a la vista 2D desde la 3D, basta con hacer clic nuevamente en el mismo ícono.

\noindent\sphinxincludegraphics{{3d-2d}.png}

\sphinxstepscope


\subsection{Herramientas para el visor}
\label{\detokenize{tools/measuring:herramientas-para-el-visor}}\label{\detokenize{tools/measuring::doc}}
\noindent{\sphinxincludegraphics{{medicion1}.png}\hspace*{\fill}}

\sphinxAtStartPar
El Visor Geoexpress cuenta con herramientas para medir distancias y áreas en el mapa. Hacemos clic en el ícono con forma de llave inglesa, lo que desplegará un submenú con cinco íconos.


\subsubsection{Medición de distancias}
\label{\detokenize{tools/measuring:medicion-de-distancias}}
\noindent{\hspace*{\fill}\sphinxincludegraphics{{medicion2}.png}}

\sphinxAtStartPar
El primer ícono sirve para medir distancias. Para iniciar una medición, haga clic en el ícono y luego en el punto del mapa desde donde comenzará a medir. Al mover el mouse desde ese punto, se trazará una línea punteada hacia el próximo punto. Una vez que el cursor llegue al punto final, haga clic nuevamente. Puede finalizar la medición haciendo doble clic, o bien trazar otra línea para continuar midiendo, aunque esta sería una nueva medición, y su longitud no se sumaría a la anterior. Si realiza varias mediciones simultáneamente, finalícelas de la misma manera, haciendo doble clic en el punto final.

\noindent\sphinxincludegraphics{{medicion3}.png}

\sphinxAtStartPar
Una vez finalizada la medición entre los puntos, aparecerán otros puntos intermedios. Al hacer clic en ellos, sin soltar el botón del mouse, podrá moverlos para realizar nuevas mediciones parciales. Si ajusta los puntos originales, también podrá modificar su posición con el zoom.

\noindent\sphinxincludegraphics{{medicion4}.png}


\subsubsection{Medición de areas}
\label{\detokenize{tools/measuring:medicion-de-areas}}
\noindent{\hspace*{\fill}\sphinxincludegraphics{{medicion5}.png}}

\sphinxAtStartPar
El segundo ícono permite medir áreas. Haga clic en el ícono y luego en uno de los vértices del área. Continúe haciendo clic en cada vértice para definir el contorno del área. Al llegar al último vértice, haga doble clic para finalizar la medición, y se mostrará la superficie del área en km².

\sphinxAtStartPar
Al igual que con la medición de distancias, después de marcar el área podrá editarla moviendo los nodos que ha creado o agregando nuevos con los puntos intermedios.

\noindent\sphinxincludegraphics{{medicion6}.png}

\sphinxAtStartPar
Para eliminar las mediciones del mapa, haga clic en el ícono de goma de borrar.

\noindent{\hspace*{\fill}\sphinxincludegraphics{{medicion7}.png}\hspace*{\fill}}


\subsubsection{Modo presentación}
\label{\detokenize{tools/measuring:modo-presentacion}}
\noindent{\hspace*{\fill}\sphinxincludegraphics{{medicion8}.png}}

\sphinxAtStartPar
El visor tiene la opción de activar el modo presentación. El usuario debe hacer clic en el anteúltimo ícono de la barra de herramientas. Este modo permite una rotación continua de la cámara en torno al centro del mapa, ocultando todos los botones y barras del visor para mostrar solo la imagen del mapa. Para salir del modo presentación, haga clic en cualquier parte de la pantalla.


\subsubsection{Impresión}
\label{\detokenize{tools/measuring:impresion}}
\sphinxAtStartPar
El último ícono permite al usuario generar un archivo imprimible a partir de la imagen del mapa. Al hacer clic en el botón, deberá configurar parámetros como tamaño de página, orientación, formato y DPI.

\sphinxAtStartPar
Podrá previsualizar el área de impresión, que aparecerá iluminada en el mapa. Las zonas más grises no se imprimirán.

\sphinxAtStartPar
Después de configurar todo, haga clic en \sphinxstylestrong{Generar} y el archivo se descargará. Para salir de la función de impresión, haga clic en la \sphinxstylestrong{x} a la derecha del encabezado del menú.

\noindent\sphinxincludegraphics{{medicion9}.png}

\sphinxstepscope


\section{Selector de mapa base}
\label{\detokenize{basemap/index:selector-de-mapa-base}}\label{\detokenize{basemap/index::doc}}
\sphinxAtStartPar
En la esquina inferior derecha de la pantalla, se encuentra un cuadrado con un mapa en su interior, con el cual es posible elegir el mapa base del visor, sobre el que se muestran todas las capas. Al pasar el cursor sobre este cuadrado, se despliegan junto a él otros 4 cuadrados con un mapa base diferente cada uno.

\noindent{\hspace*{\fill}\sphinxincludegraphics{{base1}.png}\hspace*{\fill}}

\sphinxAtStartPar
El mapa base que se encuentra por defecto en el visor es Argenmap, el mapa base del Instituto Geográfico Nacional (IGN).

\noindent\sphinxincludegraphics{{base2}.png}

\sphinxAtStartPar
Luego, está el mapa de OpenStreetMap.

\noindent\sphinxincludegraphics{{base3}.png}

\sphinxAtStartPar
Otro mapa base disponible es el Argenmap gris.

\noindent\sphinxincludegraphics{{base4}.png}

\sphinxAtStartPar
También se puede seleccionar el Argenmap oscuro como mapa base.

\noindent\sphinxincludegraphics{{base5}.png}

\sphinxAtStartPar
Por último, se halla disponible el mapa de imágenes satelitales Esri.

\noindent\sphinxincludegraphics{{base6}.png}



\renewcommand{\indexname}{Índice}
\footnotesize\raggedright\printindex
\end{document}